%----------------------------------------------------------------
%
%  File    :  available_dataset.tex
%
%  Authors : Thomas Lerchbaumer
% 
%  Created :  19 March 2022
% 
%  Changed :  19 March 2022
% 
%----------------------------------------------------------------


\chapter{What insights can be gathered?}
This section focuses on which potential information from the available dataset can be extracted and utilized for a reporting dashboard. Furthermore the dataset will be analysed for metrics that can support and improve the current yield management. 


\section{Improving the Yield Management}
In general Yield Management (YM) describes the way how limited resources like hotel rooms, seats within an air plane or available buses are assigned to customers by leveraging the highest possible revenue. American Airlines claims that by utilizing YM they are able to increase their revenue by 500 million dollars per year \cite{ym_practice}. Before integrating YM a few considerations about the characteristics of teh provided service need to be made otherwise YM might actually cause a decrease in revenue. The following characteristics are suitable for the utilization of YM:\cite{ym_practice}
\begin{itemize}
  \item Storing surplus resources can either be costly or unattainable. 
  \item Whenever future demand is uncertain commitments need to be done.
  \item It is possible to differentiate between customer segments
  \item A single unit can be used to provide different services 
  \item The company is not legally limited in their actions of selling a certain service or not.
\end{itemize}
As YM is already in place at busfinder one of the characteristics mentioned above comes along with a high uncertainty. Although commitments for uncertain future demand are made the ability to predict future bookings would further improve the YM. Being able to predict future bookings during ordinary market conditions (e.g. no travel restrictions in place) those estimates directly can be used to influence factor of capacity management - How many buses are available? This directly influences the pricing strategy because a higher demand results in a higher price.

Both Artificial Neural Networks (ANNs) and their usage for time series forecasting further evolved over the past years. Additionally libraries like tensorflow  reduce the complexity of developing ANNs. Hence this method is a suitable solution to predict future bookings. Therefore the basics of ANN and the development of two different models are explained in chapter \ref{chap:predict}.

\section{Averages}
\label{sec:averages}
As averages may seem trivial they still can provide valuable information. By comparing averages over time trends in the market become visible. However their usage should always been in combination with additional statistical measures. When analysing the dataset the following attributes could indicate market trends: 
\begin{itemize}
\item \verb|pax|
\item \verb|distanceInMeters|
\end{itemize}
By looking on the average PAX over a certain time period the metric indicates weather the number of passengers increase, decreases or stalls over time. This information along with additional statistics can assist an operator in their future planing when it comes to their fleet management. As the average in this case indicates the demand for required seats a bus should have. For example the major part of an operator's fleet are buses with 90 seats but his average PAX is around 60 which is decreasing considerations about buying smaller buses can be made. This would improve the cost-efficiency as smaller buses are cheaper, consume less petrol an maintenance costs are lower. Furthermore a decrease or increase of the average travelled distance can be used to decide weather or not electric buses might be an alternative.
As the example stated above applying the average on those parameters without any filters in place do not provide any significant information. Therefore the averages are used together with additional characteristics gathered from the data set like the grouping of certain attributes. 

\section{Grouping Data}
One part of the statistical analysis is data grouping. One reason why data is grouped is to simplify complex data structures. Furthermore it enables the possibility to summarize certain characteristics present within the data set. Another benefit of grouping data is that it might reveal potential relationships. Additionally grouping can be used to improve predictive models as demonstrated in section\ref{sec:implementation}.
Analysing the given data set reveals the following attributes offer valuable insights when grouping is applied to them: 
\begin{itemize}
\item \verb|taskFrom_lat/lng| and \verb|taskTo_lat/lng|
\item \verb|createdAt|
\item \verb|taskFrom_time|
\end{itemize}
\subsection{Geographical Grouping}
As latitude and longitude are numerical described using numerical values their usage for geographical grouping is preferred over string values which are used for attributes like \verb|taskFrom_address| and \verb|taskTo_address|. Furthermore mathematical operations on coordinates like calculate the distance between two starting points allow us to modify how the data is grouped. Therefore the geographical data is utilized to group search requests by their departure and destination location. This provides the bus operator insights about popular routes. As bus operators determine maximal travel distances to departure places depending on their logistical base this data might reveal connections with high frequency. Depending on the additional distance the operator might need and it's current utilization an operator might decide to add an exception for this specific region to allocate additional bookings. Furthermore this information could be used to influence the pricing strategy for routes with high demand.
Geographical grouping in combination with the attribute \verb|amountSearchResults| can be applied to improve the offered service. Whenever \verb|amountSearchResults| is equal to zero no buses were offered for a certain request. As the grouping might reveal high demand for certain connections bus operators could consider to supply those connections as there are no competitors in this region. This results in a higher utilization of the operators bus fleet. 
\subsection{Grouping by Date}
Grouping search requests by date reveals valuable information that can be used for marketing purposes. Grouping request on an hourly basis reveals information about daytimes with high or low user frequency. This fact can be utilized to optimize potential ad campaigns. Grouping bookings by departure date reveals dates with high demand. Furthermore this metric is used to train the prediction models described in chapter \ref{chap:predict}. As this information indicates the availability of buses on the given day. Furthermore seasonal trends become visible.\newline

%\section{Conversion Rates}




%The implementation and visualisation of topics discussed in this section are carried out in chapter \ref{chap:analytical_dashboard}

%The carried out analysis does not claim for completeness. As there are several insights that can be gathered from the requests. The data set is analysed for metrics that provide insights that support operators as well as busfinder in their decision making and to further improve their service. 