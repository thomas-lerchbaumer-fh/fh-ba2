%----------------------------------------------------------------
%
%  File    :  lit_research.tex
%
%  Authors : Thomas Lerchbaumer
% 
%  Created :  19 March 2022
% 
%  Changed :  19 March
% 
%----------------------------------------------------------------

\chapter{Info}
\label{chap:info}

\subsection{Research Question}
What Neural Networks are suitable to predict future bus bookings and which attributes of bus search requests can be utilized to create an analytical dashboard.


Research Question BA1: How to create a theoretical framework that covers basic DS
aspects and visualization techniques in order to develop an analytical web-based dashboard



\subsection{General}
Basic Idea/Content: \newline

\begin{itemize}
  \item Explain available dataset, data structure, how data is gathered 
  \item Explain what techniques are used to clean the base dataset
  \item Interdisciplinary - explain why certain KPIs or models are choosen and are applied onto the dataset 
  \item Prediction Model - explain the technologies, methods etc. used to create a ML based prediction model (To improve Yield Management). Maybe two models, Supervised Learning and unsupervised learning 
  \item Data Clustering and other KPIs + applied statistical models (e.g. Clustering, LR), Heatmaps etc. 
  \item Visualisation techniques used to display results of the applied statistical models 
  \item Maybe? Short chapter about technical setup of the Dashboard   
\end{itemize}
In general i would appreciate some general feedback what else could/should be descriped within this thesis. I think the Prediction Model will be the main aspect of this thesis (How it is created + which techniques are used, how is the performance when comparing prediction to actual booking data etc.) 




\subsection{Lit Research}
\cite{review_ml_styles} - provides also a lot of useful references to other papers that can be used 
\newline
\cite{prediction_stock_market} - ML 
\newline
\cite{hands_on_ML}, \cite{intro_ml}, \cite{tf_ctr},  \cite{bd_tf_price_forecasting} - Tensorflow, ML etc. 
\newline
\cite{kpi_imrpove_businiess}, \cite{kpi_imrpove_decision_making} - interdisciplinary to provide context which KPI's etc are chosen etc. 

