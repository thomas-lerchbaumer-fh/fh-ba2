%----------------------------------------------------------------
%
%  File    :  thesis.tex
%
%  Authors :  Keith Andrews, IICM, TU Graz, Austria
%             Manuel Koschuch, FH Campus Wien, Austria
%			  Sebastian Ukleja, FH Campus Wien, Austria
% 
%  Created :  22 Feb 96
% 
%  Changed :  14 Oct 2020
%
%  For suggestions and remarks write to: sebastian.ukleja@fh-campuswien.ac.at
% 
%----------------------------------------------------------------

% --- Setup for the document ------------------------------------

%Class for a book like style:
\documentclass[11pt,a4paper,oneside]{scrbook}
%For a more paper like style use this class instead:
%\documentclass[11pt,a4paper,oneside]{thesis}

%input encoding for windows in utf-8 needed for Ä,Ö,Ü etc..:
\usepackage[utf8]{inputenc}
%input encoding for linux:
%\usepackage[latin1]{inputenc}
%input encoding for mac:
%\usepackage[applemac]{inputenc}

\usepackage[english]{babel}
% for german use this line instead:
%\usepackage[ngerman]{babel}

%needed for font encoding
\usepackage[T1]{fontenc}
\usepackage{float}
\usepackage{array}
\usepackage{tcolorbox}



% want Arial? uncomment next two lines...
%\usepackage{uarial}
%\renewcommand{\familydefault}{\sfdefault}

%some formatting packages
\usepackage[bf,sf]{subfigure}
\renewcommand{\subfigtopskip}{0mm}
\renewcommand{\subfigcapmargin}{0mm}

%For better font resolution in pdf files
\usepackage{lmodern}

\usepackage{url}

%\usepackage{latexsym}

\usepackage{geometry} % define pagesize in more detail


\usepackage{colortbl} % define colored backgrounds for tables

\usepackage{courier} %for listings
\usepackage{listings} % nicer code formatting

\usepackage{tabularx}


\definecolor{codegreen}{rgb}{0,0.6,0}
\definecolor{codegray}{rgb}{0.5,0.5,0.5}
\definecolor{codepurple}{rgb}{0.58,0,0.82}
\definecolor{backcolour}{rgb}{0.95,0.95,0.92}

\lstset{basicstyle=\ttfamily,breaklines=true}

\usepackage{graphicx}
  \pdfcompresslevel=9
  \pdfpageheight=297mm
  \pdfpagewidth=210mm
  \usepackage[         % hyperref should be last package loaded
    pdftex, 		   % needed for pdf compiling, DO NOT compile with LaTeX
    bookmarks,
    bookmarksnumbered,
    linktocpage,
    pagebackref,
    pdfview={Fit},
    pdfstartview={Fit},
    pdfpagemode=UseOutlines,                 % open bookmarks in Acrobat
  ]{hyperref}
\DeclareGraphicsExtensions{.pdf,.jpg,.png}
\usepackage{bookmark}

\usepackage[title]{appendix}

\lstdefinestyle{mystyle}{
    backgroundcolor=\color{backcolour},   
    commentstyle=\color{codegreen},
    keywordstyle=\color{magenta},
    numberstyle=\tiny\color{codegray},
    stringstyle=\color{codepurple},
    basicstyle=\ttfamily\footnotesize,
    breakatwhitespace=false,         
    breaklines=true,                 
    captionpos=b,                    
    keepspaces=true,                 
    numbers=left,                    
    numbersep=5pt,                  
    showspaces=false,                
    showstringspaces=false,
    showtabs=false,                  
    tabsize=2
}
\lstset{style=mystyle}

%paper format
\geometry{a4paper,left=30mm,right=25mm, top=30mm, bottom=30mm}

% --- Settings for header and footer ---------------------------------
\usepackage{scrlayer-scrpage}
\clearscrheadfoot
\pagestyle{scrheadings}
\automark{chapter}

%Left header shows chapter and chapter name, will not display on first chapter page use \ihead*{\leftmark} to show on every page
\ihead{\leftmark} 	
%\ohead*{\rightmark}	%optional right header
\ifoot*{Thomas Lerchbaumer}		%left footer shows student name
\ofoot*{\thepage}		%right footer shows pagination
%---------------------------------------------------------------------

%Start of your document beginning with title page
\begin{document}


% --- Main Title Page ------------------------------------------------
\begin{titlepage}
\frontmatter

\begin{picture}(50,50)
\put(-70,40){\hbox{\includegraphics{images/logo.png}}}
\end{picture}

\vspace*{-5.8cm}

\begin{center}

\vspace{6.2cm}

\hspace*{-1.0cm} {\LARGE \textbf{Data Science and Visualisation Techniques applied on Bus Search Requests and its Correlating Booking Data\\}}
\vspace{0.2cm}
\hspace*{-1.0cm} Subtitle\\

\vspace{2.0cm}

\hspace*{-1.0cm} { \textbf{Bachelor Thesis\\}}

\vspace{0.65cm}

\hspace*{-1.0cm} Submitted in partial fulfillment of the requirements for the degree of \\

\vspace{0.65cm}

\hspace*{-1.0cm} \textbf{Bachelor of Science in Engineering\\}

\vspace{0.65cm}

\hspace*{-1.0cm} to the University of Applied Sciences FH Campus Wien \\
\vspace{0.2cm}
\hspace*{-1.0cm} Bachelor Degree Program: Computer Science and Digital Communications \\

\vspace{1.6cm}

\hspace*{-1.0cm} \textbf{Author:} \\
\vspace{0.2cm}
\hspace*{-1.0cm} Thomas Lerchbaumer \\

\vspace{0.7cm}

\hspace*{-1.0cm} \textbf{Student identification number:}\\
\vspace{0.2cm}
\hspace*{-1.0cm} 2010475111 \\

\vspace{0.7cm}

\hspace*{-1.0cm} \textbf{Supervisor:} \\
\vspace{0.2cm}
\hspace*{-1.0cm} René Goldschmid, MSc \\

\vspace{0.7cm}

% Reviewer if needed
%\hspace*{-1.0cm} \textbf{Reviewer: (optional)} \\
%\vspace{0.2cm}
%\hspace*{-1.0cm} Title first name surname \\


\vspace{1.0cm}

\hspace*{-1.0cm} \textbf{Date:} \\
\vspace{0.2cm}
\hspace*{-1.0cm} 14.08.2023 \\

\end{center}
\end{titlepage}

\newpage

\vspace*{16cm}
\setcounter{page}{1}

% --- Declaration of authorship ------------------------------------------
\hspace*{-0.7cm} \underline{Declaration of authorship:}\\\\
I declare that this Bachelor Thesis has been written by myself. I have not used any other than the listed sources, nor have I received any unauthorized help.\\\\
I hereby certify that I have not submitted this Bachelor Thesis in any form (to a reviewer for assessment) either in Austria or abroad.\\\\
Furthermore, I assure that the (printed and electronic) copies I have submitted are identical.
\\\\\\
Date: \hspace{6cm} Signature:\\

% --- English Abstract ----------------------------------------------------
\cleardoublepage
\chapter*{Abstract}
As online travel booking platforms continuously grow, knowledge of customer behaviour is crucial. Hence this thesis aims to uncover patterns, trends, and correlations in users' search behaviours, which can lead to improved decision making for both travellers and service providers. In order to do so bus search requests as well as its correlating bookings are analysed to improve the offered service from busfinder, an online bus booking platform. The company’s available booking data is processed to create a prediction model to forecast future bus utilisation by means of an experiment. The predictions are used to enhance the current yield management of busfinder. Moreover, a literature review is conducted to set the theoretical framework regarding neural networks and its applicability in a production environment. The analysis of bus search requests revealed grouping possibilities based on various attributes. Firstly, grouping search requests based on their departure and destination areas routes with high utilisation become visible. This knowledge can be used to influence an operator’s pricing strategy. Secondly, grouping search requests on an hourly basis revealed demand patterns throughout the day. By grouping search requests on a daily basis seasonal patterns become available. Both groupings can be used to improve the scheduling of ad campaigns and to improve the pricing strategy. Thirdly, trends about future demand become visible by providing averages gathered from the attributes pax and travel distance in a timely context. Visualising the gathered insights allow bus operators to analyse the data in real time and to use it during a data driven decision making process. Overall, the obtained knowledge can be used to improve the offered service as well as to maximise the potential profit from certain routes.
% --- German Abstract ----------------------------------------------------
\cleardoublepage
\chapter*{Kurzfassung}

Da Online-Reisebuchungsplattformen ständig wachsen, ist das Wissen über das Kundenverhalten von entscheidender Bedeutung. Daher zielt diese Arbeit darauf ab, Muster, Trends und Zusammenhänge im Suchverhalten der Nutzer aufzudecken. Die gewonnenen Erkenntnisse sollen Entscheidungsfindung sowohl für Reisende als auch für Dienstleistungsanbieter verbessern. Zu diesem Zweck werden Bussuchanfragen und die damit verbundenen Buchungen analysiert, um den angebotenen Service von busfinder, einer Online-Busbuchungsplattform, zu verbessern. Die verfügbaren Buchungsdaten des Unternehmens werden verarbeitet, um ein Prognosemodell zu erstellen, um die zukünftige Busauslastung vorherzusagen. Die Vorhersagen werden verwendet, um das aktuelle Yield Management von busfinder zu verbessern. Darüber hinaus wird eine Literaturrecherche durchgeführt, um den theoretischen Rahmen für neuronale Netze und ihre Anwendbarkeit in einer Produktionsumgebung festzulegen. Die Analyse der Suchanfragen ergab Gruppierungsmöglichkeiten für verschiedener Attribute. Erstens werden durch die Gruppierung von Suchanfragen nach Abfahrts- und Zielgebieten Strecken mit hoher Auslastung sichtbar. Dieses Wissen kann genutzt werden, um die Preisstrategie eines Betreibers zu verbessern. Zweitens zeigt die Gruppierung von Suchanfragen auf stündlicher Basis Nachfragemuster über den Tag hinweg. Durch die Gruppierung von Suchanfragen auf Tagesbasis werden saisonale Muster sichtbar. Beide Gruppierungen können zur Verbesserung für die Planung von Werbekampagnen und zur Verbesserung der Preisstrategie verwendet werden. Drittens werden Trends in Bezug auf die künftige Nachfrage sichtbar, indem Durchschnittswerte aus den Attributen Pax und Reisedistanz in einem zeitlichen Kontext bereitgestellt werden. Die Visualisierung der gewonnenen Erkenntnisse ermöglicht es den Busbetreibern, die Daten in Echtzeit zu analysieren und im Rahmen eines datengesteuerten Entscheidungsfindungsprozesses zu nutzen. Insgesamt können die gewonnenen Erkenntnisse genutzt werden, um den angebotenen Service zu verbessern und den potenziellen Gewinn auf bestimmten Strecken zu maximieren.



% --- Abbrevations ----------------------------------------------------
\chapter*{List of Abbreviations}
\vspace{0.65cm}

\begin{table*}[htbp]
		\begin{tabular}{ll}
			BIM & Business Indicator Management \\
			BP & Backpropagation \\
			CNN & Convolutional Neural Network \\
			LSTM & Long Short-Term Memory \\
			KPI & Key Performance Indicator\\
			MAE & Mean Absolute Error \\
			ML  &  Machine Learning\\
			NN & Neural Networks \\
			RNN & Recurrent Neural Network \\
			YM & Yield Management\\
		\end{tabular}
\end{table*}

% --- Key terms ----------------------------------------------------
\newpage
\chapter*{Key Terms}
\vspace{0.65cm}

\begin{itemize}
	\setlength{\itemsep}{0pt}
	\item[] Convolutional Neural Network
	\item[] Data Science
	\item[] Long Short Term Network
	\item[] Neural Networks
	\item[] Time Series Forecasting
	\item[] Yield Management
\end{itemize}

% --- Table of contents autogenerated ------------------------------------
\newpage
\tableofcontents
\thispagestyle{empty}

% --- Begin of Thesis ----------------------------------------------------
\mainmatter
%----------------------------------------------------------------
%
%  File    :  introduction.tex
%
%  Authors : Thomas Lerchbaumer
% 
%  Created :  19 March 2022
% 
%  Changed :  19 March
% 
%----------------------------------------------------------------

\chapter{Introduction}
\label{chap:introduction}
The rise of machine learning, and especially Neural Networks, has created so far unheard-of capabilities for predictive analysis in the context of individual traffic in the modern transportation landscape. As digitalization emerged rapidly over the past years the possibilities for data-driven decision making further evolved. 
%%----------------------------------------------------------------
%
%  File    :  lit_research.tex
%
%  Authors : Thomas Lerchbaumer
% 
%  Created :  19 March 2022
% 
%  Changed :  19 March
% 
%----------------------------------------------------------------

\chapter{Info}
\label{chap:info}

\subsection{Research Question}
What Neural Networks are suitable to predict future bus bookings and which attributes of bus search requests can be utilized to create an analytical dashboard.


Research Question BA1: How to create a theoretical framework that covers basic DS
aspects and visualization techniques in order to develop an analytical web-based dashboard



\subsection{General}
Basic Idea/Content: \newline

\begin{itemize}
  \item Explain available dataset, data structure, how data is gathered 
  \item Explain what techniques are used to clean the base dataset
  \item Interdisciplinary - explain why certain KPIs or models are choosen and are applied onto the dataset 
  \item Prediction Model - explain the technologies, methods etc. used to create a ML based prediction model (To improve Yield Management). Maybe two models, Supervised Learning and unsupervised learning 
  \item Data Clustering and other KPIs + applied statistical models (e.g. Clustering, LR), Heatmaps etc. 
  \item Visualisation techniques used to display results of the applied statistical models 
  \item Maybe? Short chapter about technical setup of the Dashboard   
\end{itemize}
In general i would appreciate some general feedback what else could/should be descriped within this thesis. I think the Prediction Model will be the main aspect of this thesis (How it is created + which techniques are used, how is the performance when comparing prediction to actual booking data etc.) 




\subsection{Lit Research}
\cite{review_ml_styles} - provides also a lot of useful references to other papers that can be used 
\newline
\cite{prediction_stock_market} - ML 
\newline
\cite{hands_on_ML}, \cite{intro_ml}, \cite{tf_ctr},  \cite{bd_tf_price_forecasting} - Tensorflow, ML etc. 
\newline
\cite{kpi_imrpove_businiess}, \cite{kpi_imrpove_decision_making} - interdisciplinary to provide context which KPI's etc are chosen etc. 


%----------------------------------------------------------------
%
%  File    :  theory_lstm.tex
%
%  Authors : Thomas Lerchbaumer
% 
%  Created :  19 March 2022
% 
%  Changed :  19 March
% 
%----------------------------------------------------------------



\chapter{Theoretical Basis}
\label{chap:theory_basis}
As this paper aims at evaluation various NNs regarding their usage for time series forecasting this chapter explains which NNs are suitable and how they can be integrated. The knowledge of potential future bookings provide useful insights when it comes to YM. YM in general describes controlling price and capacity control in a simultaneous ways \cite{yield_m}. Therefore those predictions can be used to support bus operators in their pricing strategy. This chapter focuses on creating two prediction models utilising different techniques based on the data that is available. 
Both models are implemented using Python and the following libraries:
\begin{itemize}
\item  \verb|matplotlib|\footnote{https://matplotlib.org/} - used for plotting
\item \verb|pandas|\footnote{https://pandas.pydata.org/} - used for data manipulation 
\item \verb|tensorflow|\footnote{https://www.tensorflow.org/} - provides ML models
\item \verb|keras|\footnote{https://keras.io/} - Neural Network library
\end{itemize}

As there are various models available a literature review is conducted to figure out which models fit the purpose of time series forecasting. It turns out that the most promising NNs that can be utilised for time series prediction are either Convolutional Neural Networks (CNN) or RNN especially LSTM\cite{nn_1}\cite{nn_2}\cite{lstm_1}\cite{lstm_2}\cite{rnn_time_series_predict}.

\section{Backpropagation}
\label{sec:bp}

As both of the models LSTM and CNN use Backpropagation (BP) for its training the basic concepts of the algorithm are explained in this section. To understand the logic of BP a few terms need a detailed explanation: 
\subsubsection{Gradient}
The gradient also called gradient descent is a technique utilised to optimise the loss function used within BP. 
This implies that the gradient descent shows how the weights and biases should be changed in order to decrease the actual error value, which is the outcome of the applied loss function.
 \cite{bp_basic}
\subsubsection{Bias}
The bias is an additional parameter used in each neuron of a NN. It is used to directly influence the activation function to offset the results either to the negative or positive direction \cite{bias}. When looking at the sigmoid function without any bias in place where \verb|x| correlates to the input value and \verb|w| indicates the used weight: 
\begin{equation}
 \sigma(x) = \frac{1} {1 + e^{-{(w*x)}}}
\label{eq:eq_4}
\end{equation}
When looking at figure \ref{fig:sig_wo_bias} the weights only influence the steepness of the function but will not shift it along the x axis.
\begin{figure}[H]
	\centering
		\includegraphics[width=8cm]{images/sigmoid_w_weights}
	\caption{Sigmoid function with different weights and no bias - [source:\cite{lstm_module}]}
	\label{fig:sig_wo_bias}
\end{figure}
 To shift the function along the x axis the sigmoid function is adapted with the bias \verb|b| value: 
\begin{equation}
 \sigma(x) = \frac{1} {1 + e^{-{(w*x+b)}}}
\label{eq:eq_4}
\end{equation}
By adding the bias value as constant to the sigmoid function can be shifted along the x axis as shown in figure \ref{fig:sig_w_bias}.
\begin{figure}[H]
	\centering
		\includegraphics[width=8cm]{images/bias}
	\caption{Sigmoid function with same weights but different bias - [source:author]}
	\label{fig:sig_w_bias}
\end{figure}
The bias therefore is utilised to directly influence the result of the activation function and whether or not a certain neuron is activated or not. 
\subsubsection{Activation Function}
Activation functions introduce non-linearity characteristics to NN. This function is applied to the output of a neuron and decides whether or not a neuron is activated or not.\cite{activation} In combination with the descent gradient this function enables the NN to learn complex patterns within a training set.
The sigmoid function \cite{bp_basic} was first employed as the activation function for NN, but as of today, several additional functions, including softmax, Tanh, and ReLu, have evolved \cite{activation}.

\subsubsection{Phases} 
Backpropagation follows an iterative process. At the beginning there is the feed forward pass. During this phase the input data is passed through all layers. Each hidden layer applies a linear function to create certain weights those outputs then are fed to the activation function. The output of the activation function is utilised as the input for the following neuron, depending on how many hidden layers are employed in the NN. At the end the predicted outputs by the model are compared with the actual outputs of the training data. This comparison is evaluated through a loss function. At this step the actual learning process of the model starts by computing the gradient of the loss function based on the output of the NN. After that the back pass is initiated. Along this phase the gradients of each previous layer are multiplied with a local gradient and its weights which results in a gradient in respect to the layer's input. After receiving all gradients  based on the network's input data, all weights and biases are updated and optimised to reduce the result of the loss function. The steps forward pass, loss calculation, back pass, and the updates of weights and biases are repeated to improve the network in an iterative way. \cite{bp_basic} Figure \ref{fig:bp} demonstrates the logic of the backpropagation algorithm. Whereas \verb|wu| represents the updated weights after each iteration.  

\begin{figure}[H]
	\centering
		\includegraphics[width=14cm]{images/bp}
	\caption{Simplified logic of the backpropagation algorithmus - [source:\cite{bp_basic}]}
	\label{fig:bp}
\end{figure}
\section{The Models}
Time series forecasting can be performed using CNN and RNN/LSTM models. To create accurate prediction models a basic knowledge of the models functionalities are required. Therefore this section explains the components of each NN as well as the approaches those models follow. 

\subsection{LSTM}
\label{sec:lstm}
LSTM is a RNN and was invented by \cite{lstm_inventor} in 1997. Until today this NN is widley used for time series forecasting and provides reliable results for short as well as long term predictions \cite{rnn_moharm}. LSTM has so called memory cells which are responsible to store the state of data. Whenever information arrives at a memory cell its outcome is defined by refreshing the cell state with the newly arrived information. LSTM utilizes gates to control a cells state by either including or excluding information \cite{lstm_stock}. The gates are called: 
\begin{itemize}
\item input gate - data selection and storage for upcoming state
\item forget gate - data selection and storage which will not be used for the upcoming state
\item output gate - sets information within the state that is send to the output
\end{itemize}
Those gates are created by combining sigmoid functions. The results of these gates are values ranging from zero to one. A result of zero indicates the cell to not pass any information whereas values close to one indicate the cell to pass all information. 
The LSTM module or Repeating module consist of four NN layers which interact together as shown in Figure \ref{fig:lstm_rep_model}:
\begin{figure}[H]
	\centering
		\includegraphics[width=14cm]{images/lstm_module}
	\caption{Repeating LSTM module - [source:\cite{lstm_module}]}
	\label{fig:lstm_rep_model}
\end{figure}
In total the repeating model has three gate activation functions which are named $\sigma_1$, $\sigma_2$,  $\sigma_3$ and shown in figure \ref{fig:lstm_rep_model}. Furthermore $\sigma_1$ and $\sigma_2$ act as output activation functions too. The cell state is illustrated using a blue line which starts at St-1 and indicates the previous memory block to St representing the current memory block. The amount of information that is passed is regulated by layer  $\sigma_1$ using the following function:
\begin{equation}
cf_t = \sigma_1 (W_cf * [O_t-1, x_t] + b_cf)
\cite{lstm_module}
\label{eq:eq_1}
\end{equation}
To store fresh information to the cell state, two network layers are employed. Therefore sigmoid layer $\sigma_2$ chooses the values which are updated by utilising the following formula:
\begin{equation}
l_t = \sigma_2(W_1 *[O_t-1, x_t]+ b_l)
\cite{lstm_module}
\label{eq:eq_2}
\end{equation}
Layer $\phi_1$ or \verb|tanh| is created by using new candidate values. This layer outputs a  vector by utilzing the following formular: 
\begin{equation}
\widetilde{S}_t = tanh(W_s * [O_t-1, X_t] + b_s)
\cite{lstm_module}
\label{eq:eq_3}
\end{equation}
The last step includes combination of both states \ref{eq:eq_2} and \ref{eq:eq_3} which is added to the state. Also the state is reconditioned by applying: \cite{lstm_module}
\begin{equation}
S_t = cf_t * S_t1 + I_t * \widetilde{S}_t-1
\cite{lstm_module}
\label{eq:eq_4}
\end{equation}

The reason why a LSTM model is used for this purpose is that a standalone RNN is challenging to train due to its characteristics. As BP is used for RNNs, problems like vanishing-gradient can occur. The gradient in general can be understood as a computed value through all time steps which in the end used to update parameters of the RNN. The vanishing-gradient over time results in information decay. By implementing a LSTM module this problem can be solved. \cite{lstm_overcome_rnn_problem}

\subsubsection{Bidirectional LSTM}
Bidirectional LSTMs are able to look in both directions past and future. This is achieved by processing the available data into both directions. Therefore those models make use of bidirectional layers. Those layers split up the used neurons into two directions. \cite{bi_di_1} This provides more information to the network as the model is now capable of storing the forward state as well as the backward state, resulting in potentially more accurate results \cite{bi_di_2}. 

\subsection{CNN}
\label{sec:cnn}
CNNs follow the concept of NN consisting of multiple layers. The scope of application for this kind of network reaches from computer vision problems to time-series forecast modelling. Whereas data provided for image classification is structured in multi dimensional arrays (matrices), data used for time-series forecasting is provided via one dimensional arrays.\cite{cnn_intro} A CNN provides different types of layers. Those layer types are called Pooling Layer (PL), Fully Connected (FC), Convolution layer (CL) and Flatten layer (FL). The connection of those layers are demonstrated in figure \ref{fig:cnn_struct}.
\begin{figure}[H]
	\centering
		\includegraphics[width=10cm]{images/1d_cnn_model}
	\caption{One Dimensional CNN Structure - [source:\cite{cnn_vechicle}]}
	\label{fig:cnn_struct}
\end{figure}

CNN is based on convolution which is a linear operation that multiplies input data with convolution filters. Those filters which are also called kernels correlate to a set of weights \cite{cnn_vechicle}. The kernel values are created during the learning process and are optimised from the NN utilising BP. Furthermore this layer is utilised to detect features within the given one dimensional array. Those features are stored within a feature map and are calculated by applying convolutions on the input data. One crucial parameter to detect proper features is the size of the kernel. The kernel size can be understood as a number of weights that are multiplied with the input data. After each multiplication the sequence is shifted along the input data. Each shift during this process produces one output which is stored in the feature map. The following example demonstrates how this process is done:\cite{1d_cnn}
\begin{lstlisting}
Input Data: [4,7,10,43,20,10] e.g. number of bookings per day. 
Kernel: [0.5,0.25,0.2] Kernel size = 3
1st Multiplication: 4*0.5 + 7*0.25 + 10*0.2 = 5.75
2nd Multiplication: 7*0.5 + 10*0.25 + 43*0.2 = 14.6
...
Output sequence [5.75, 14.6, ...]
\end{lstlisting}
The second operation that is used within the CNN is called activation function. This non-linear function is utilised to detect complex relationships between variables and are applied onto the feature map. As of today multiple functions like ReLU, sigmoid and softmax can be used as activation functions \cite{cnn_basic3}.

The pooling layer as shown in figure \ref{fig:cnn_struct} is deployed within a NN to diminish the size of feature maps. To reduce the size pooling operations like average pooling, max pooling or sum pooling can be applied. Applying one of those operations results in less computational effort.\cite{cnn_basic}

%The activation function is also part of the fully connected layer. This layer applies the activation function onto the feature map and enables the model in combination with BP to learn complex connections between features. Furthermore this layer operates on the already flatten feature map and outputs a 2d vector.\cite{1d_cnn}

The FL converts two dimensional input data into one dimensional input vectors. Its output is used to provide values to the FC layer. Its output is used to provide values to the fully connected layer. Overfitting can be caused whenever all features are used in the flatten layer. Therefore a dropout layer can be set in place.\cite{1d_cnn} This layer cancels out neurons during the training process of a NN which reduces the model's size. 

\section{Loss Function}
\label{sec:loss_func}
The loss function is one crucial element as it evaluates the accuracy of the produced outputs from a NN. This is achieved by calculating the divergence between the predicted value and the actual value provided by the test dataset. Supervised learning deals with two different problems which is either a classification problem e.g. is the animal on the picture a cat or with regression problems like predicting future bookings. Both of those problems use a different loss function.\cite{loss_func} As this section focuses on solving a regression problem a brief overview of available loss functions and their characteristics are given. 

\begin{table*}[htbp]
	\centering
		\begin{tabularx}{\textwidth}{|l|X|}
		\hline
		\rowcolor[gray]{0.9}
		Function & Characteristics \\
		\hline
		Square loss &Sensitive to outliers (Model tends to focus on those outliers whereas accuracy for normal values decline)\\
		 \hline
		Absolute loss & Outliers do not influence the model as  severe as compared to square loss.  \\
		\hline
		Huber & Combination of square loss and absolute loss - Outliers do not influence the accuracy of results and learning from smaller errors can still be done in a efficient way  \\
		\hline
		Log-cosh & Similar to Huber when it comes to its characteristics. Does not handle large errors well because the gradient tends to stay constant. \\
		\hline
		Quantile loss & Extends absolute loss and provides prediction intervals. Utilising a punishment system for overestimated and underestimated samples. \\
		\hline
		$\epsilon$-insensitive & Focuses on samples with large prediction errors \\
		\hline	
		\end{tabularx}
	\label{tab:loss_function}
	\caption{Loss functions and their characteristics - [source:\cite{loss_func}]}
\end{table*}

By looking at the characteristics of the augmented data set shown in figure \ref{fig:augmented_data} it is clear to see that the dataset itself has got outliers repeating themselves every year. To avoid a strong focus on those peaks both models are initially trained utilising the Adam loss function.

\section{Optimise Function}
\label{sec:optimize_func}
To optimise the parameters of a NN an optimiser function is required. This function updates parameters like weights based on the results provided by the loss function \cite{optimizer}. Since both NNs described in this section make use of BP for their training. A literature review was conducted to figure out which optimiser is keen to deliver the most accurate results. By inspecting the advantages and disadvantages proposed by these works \cite{optimizer}\cite{otimizer_1}\cite{optimizer_2} it turns out that the algorithm Adam is a valid choice. This algorithm is characterised by achieving faster convergence compared to other algorithms. Furthermore Adam provides a decent performance for datasets with meager features.
Meager features can also lead to problems like underfitting and overfitting therefore those topics are discussed next. 

\section{Overfitting and Underfitting}
One problem that can occur when utilising NNs for predictions is overfitting or underfitting of the training data. Both scenarios result in a poor performance of the trained model. \cite{fitting} 
\subsection{Overfitting}

Overfitting describes the phenomenon that the model is not able to improve its problem solving capabilities after a certain period of training. There are multiple reasons for the occurrence of overfitting. One reason for example is an inaccurate or unbalanced training set. This leads to the fact that the NN produces wrong connections during its training.  Whereas the results for the training set are accurate the problems  occur during the validation phase because the model learned wrong characteristics. \cite{fitting} 
\subsection{Underfitting}
On the other hand, underfitting arises when the model cannot identify the traits of the training set and therefore struggles to achieve matching its target values. This results in high loss values. Reasons for underfitting are caused by a lack of trainable parameters as well as a NN model with a simple architecture in terms of hidden layers. \cite{fitting} 
\newline
\newline
As the data was affected during the Covid19 pandemic data augmentation which is applied during section \ref{sec:data_aug} is necessary in order to avoid both scenarios.
\newline
\newline
The basics explained during this chapter are crucial when it comes to improving a model's performance as all of the components mentioned above influence a models capability to predict future bookings. Another key element to achieve acceptable results is the utilised data itself. Therefore chapter \ref{chap:available_dataset} describes the available data, its characteristics and how the data is prepared so it can be used for time series forecasting. 
 

%----------------------------------------------------------------
%
%  File    :  available_dataset.tex
%
%  Authors : Thomas Lerchbaumer
% 
%  Created :  19 March 2022
% 
%  Changed :  19 March 2022
% 
%----------------------------------------------------------------

\chapter{Available Dataset}
\label{chap:available_dataset}
This chapter focuses on explaining and analysing the available data. The data is analyzed for Business Intelligence (BI) purposes as well as on metrics that can be used to create predictions. Whereas BI \cite{9325610} focuses on historical data and aims to support managers to make decisions traditional methods like predictive analytic asses potential future scenarios using advanced statistical methods \cite{9671997}.

\section{Data Origin}
The available dataset is gathered from a website that provides a service to find and book buses for individual journeys. This service is currently available in Austria, Germany, Switzerland and Lichtenstein. The buses itself are offered in real time by various different bus companies. Offers can vary in price which is based bus calculations which may vary from operator to operator. The data is stored in a relational database. Since the service also provides the possibility to directly book a bus, booking and corresponding user meta data is available as well. \newline




\section{Data Structure}
The service launched in March 2017 therefore booking data is available back to this date. Tracking the search requests was introduced in October 2020. 
The request table itself keeps track of 40 attributes but not all of them host valuable information that could be analysed therefore only the ones which can be analysed are listed and explained below:

\begin{itemize}
  \item \verb|task_id| - PK (incremented value) 
  \item \verb|createdAt| - At which time the search request was made.
  \item \verb|accountId| - Not empty when the user is currently logged in
  \item \verb|amountSearchResults| - How many buses can be offered
  \item \verb|containsTripCompany| - If the user wants to stop at a certain company during the trip
  \item \verb|distanceInMeters|  - Distance between departure and destination place
  \item \verb|durationInSeconds| - Duration of the trip
  \item \verb|pax|  - Amount of passengers
  \item \verb|taskFrom_address| - Departure address 
  \item \verb|taskFrom_lat and lng| - Latitude and Longitude of the departure
  \item \verb|taskTo_address|  - Destination address
  \item \verb|taskTo_lat and lng| - Latitude and Longitude of the destination
  \item \verb|taskFrom_Time|  - Desired departure time 
  \item \verb|taskTo_Time|  - Estimated arrival time
  \item \verb|cheapestPrice_amount| - The cheapest price for a bus
  \item \verb|bus_id| - The operator with the cheapest bus
  \item \verb|city| - From which city the request was made
  \item \verb|country| - From which country the request was made
\end{itemize}

Whenever a booking is made the correlating data is stored within a booking table. As the booking table contains sensitive data which is not scope of the analysis, so only three attributes are used: 

\begin{itemize}
	\item \verb|createdAt| - At which time the booking was made.
	\item \verb|company_id| - FK used to identify who received the booking
	\item \verb|task_id| - FK used to link the booking to an search request 
\end{itemize} 

\section{Data Cleansing}
\label{sec:data_cleansing}
During this process the available data is investigated for irregularities that cause distortion when applying statistical models. 

Search requests are tracked whenever a user opens the service and searches for a certain connection. Given that behaviour it may occur that a user searches for the same connection within a short time window. This behaviour results in the need of de-duplication to avoid bias. To filter out duplicates the attributes \verb|ipHash|, \verb|createdAt|, \verb|taskFrom_address| and \verb|taskTo_address|. A search request is considered as non duplicate whenever the timespan between equal entries is larger than one hour. 
To pre-process the data the following logic is applied once //todoChange: 
\begin{lstlisting}
    query = '''
    DELETE t1
    FROM search_requests_clean t1
    INNER JOIN search_requests_clean t2
        ON t1.taskFrom_address = t2.taskFrom_address
        AND t1.taskTo_address = t2.taskTo_address
        AND t1.ipHash = t2.ipHash
        AND t1.createdAt > t2.createdAt
        AND t1.createdAt - t2.createdAt <= %s
'''

    timespan = 3600  # 3600 seconds  - 1 hour
    cursor = connection.cursor()
    cursor.execute(query, (timeframe,))
    connection.commit()
\end{lstlisting}

//todo more explanation
The logic above compares all entries based on the attributes mentioned above removes equal entries that are within a timespan of 1 hour. 

Regarding validation and norming the available data present in both tables no actions are required due to fact that attributes that do not meet their defined data types are not stored in first place. 

\section{Data Augmentation}
\label{sec:data_aug}
Starting from March 2020 countries like Austria, Germany, Switzerland and Lichtenstein had to put travel restrictions into effect due to the ongoing Covid19 pandemic \\citeHere. This travel restrictions impacted the gathered booking data as those restrictions forbid travelling.
\begin{figure}[H]
	\centering
		\includegraphics[width=14cm]{images/no_augmentation}
	\caption{Bookings over time - [source:author]}
	\label{fig:noAug}
\end{figure}
Figure \ref{fig:noAug} highlights the drop of bookings starting from March 2020 until June 2022. To achieve reliable results when utilizing this data for a time series forecasting ML model this time period needs augmentation. When analysing the chart \ref{fig:noAug} an continuous growth of bookings is visible until 2021. One way to augment the data \\citehere is calculate the average growth during this time span. To substitute the distorted data the current data is replaced by the value of the previous year. This value is then multiplied by the average growth. Furthermore missing timestamps throughout the whole time series are added with a value of zero. The following logic is applied to the data frame: 

\begin{lstlisting}
df = db.get_booking_data()
average_growth = df['bookings'].pct_change().mean()
substitute_corona = pd.date_range(start='2020-03-01', end='2022-05-01', freq='D')
df['date(createdAt)'] = pd.to_datetime(df['date(createdAt)'])
df = (df.set_index('date(createdAt)')
      .reindex(pd.date_range('2018-01-01', '2023-05-01', freq='D'))
      .rename_axis(['date(createdAt)'])
      .fillna(0)
      .reset_index())

df.set_index('date(createdAt)', inplace=True)

for date in substitute_corona:
    year_ago = str(date - relativedelta(years=1)).split(" ")[0]
    val = int(math.ceil(df.loc[year_ago]['bookings'] * (1+average_growth)))
    df.loc[str(date).split(" ")[0]] = val
\end{lstlisting}

The average growth per anno is around 30\%. After applying the logic the data set looks the following:  
\begin{figure}[H]
	\centering
		\includegraphics[width=14cm]{images/with_augmentation}
	\caption{Augmented Data Set - [source:author]}
	\label{fig:noAug}
\end{figure}
The impact of this augmentation in terms of prediction accuracy is compared in chapter \ref{chap:prediction_model}.

\section{What insights can be gathered?}
This section focuses on which potential information from the available dataset can be extracted and utilized for a reporting dashboard. Furthermore the dataset will be analysed for metrics that can support and improve the current yield management. 


\subsection{Improving the Yield Management}
In general Yield Management (YM) describes the way how limited resources like hotel rooms, seats within an air plane or available buses are assigned to customers by levearging the highest possible revenue. American Airlines claims that by utilizing YM they are able to increase their revenue by 500 million dollars per year \cite{ym_practice}. Before integrating YM a few considerations about the characteristics of teh provided service need to be made otherwise YM might actually cause a decrease in revenue. The following characteristics are suitable for the utilization of YM:\cite{ym_practice}
\begin{itemize}
  \item Storing surplus resources can either be costly or unattainable. 
  \item Whenever future demand is uncertain commitments need to be done.
  \item It is possible to differentiate between customer segments
  \item A single unit can be used to provide different services 
  \item The company is not legally limited in their actions of selling a certain service or not.
\end{itemize}
As YM is already in place at busfinder one of the characteristics mentioned above comes along with a high uncertainty. Although commitments for uncertain future demand are made the ability to predict future bookings would further improve the YM. Being able to predict future bookings during ordinary market conditions (e.g. no travel restrictions in place) those estimates directly can be used to influence factor of capacity management - How many buses are available? This directly influences the pricing strategy because a higher demand results in a higher price.

Both Artificial Neural Networks (ANNs) and their usage for time series forecasting further evolved over the past years. Additionally libraries like tensorflow  reduce the complexity of developing ANNs. Hence this method is a suitable solution to predict future bookings. Therefore the basics of ANN and the development of two different models are explained in chapter \ref{chap:predict}.



%----------------------------------------------------------------
%
%  File    :  available_dataset.tex
%
%  Authors : Thomas Lerchbaumer
% 
%  Created :  19 March 2022
% 
%  Changed :  19 March 2022
% 
%----------------------------------------------------------------


\chapter{What insights can be gathered?}
This section focuses on which potential information from the available dataset can be extracted and utilized for a reporting dashboard. Furthermore the dataset will be analysed for metrics that can support and improve the current yield management. 


\section{Improving the Yield Management}
In general Yield Management (YM) describes the way how limited resources like hotel rooms, seats within an air plane or available buses are assigned to customers by leveraging the highest possible revenue. American Airlines claims that by utilizing YM they are able to increase their revenue by 500 million dollars per year \cite{ym_practice}. Before integrating YM a few considerations about the characteristics of teh provided service need to be made otherwise YM might actually cause a decrease in revenue. The following characteristics are suitable for the utilization of YM:\cite{ym_practice}
\begin{itemize}
  \item Storing surplus resources can either be costly or unattainable. 
  \item Whenever future demand is uncertain commitments need to be done.
  \item It is possible to differentiate between customer segments
  \item A single unit can be used to provide different services 
  \item The company is not legally limited in their actions of selling a certain service or not.
\end{itemize}
As YM is already in place at busfinder one of the characteristics mentioned above comes along with a high uncertainty. Although commitments for uncertain future demand are made the ability to predict future bookings would further improve the YM. Being able to predict future bookings during ordinary market conditions (e.g. no travel restrictions in place) those estimates directly can be used to influence factor of capacity management - How many buses are available? This directly influences the pricing strategy because a higher demand results in a higher price.

Both Artificial Neural Networks (ANNs) and their usage for time series forecasting further evolved over the past years. Additionally libraries like tensorflow  reduce the complexity of developing ANNs. Hence this method is a suitable solution to predict future bookings. Therefore the basics of ANN and the development of two different models are explained in chapter \ref{chap:predict}.

\section{Averages}
As averages may seem trivial they still can provide valuable information. By comparing averages over time trends in the market become visible. However their usage should always been in combination with additional statistical measures. When analysing the dataset the following attributes could indicate market trends: 
\begin{itemize}
\item \verb|pax|
\item \verb|distanceInMeters|
\end{itemize}
By looking on the average PAX over a certain time period the metric indicates weather the number of passengers increase, decreases or stalls over time. This information along with additional statistics can assist an operator in their future planing when it comes to their fleet management. As the average in this case indicates the demand for required seats a bus should have. For example the major part of an operator's fleet are buses with 90 seats but his average PAX is around 60 which is decreasing considerations about buying smaller buses can be made. This would improve the cost-efficiency as smaller buses are cheaper, consume less petrol an maintenance costs are lower. Furthermore a decrease or increase of the average travelled distance can be used to decide weather or not electric buses might be an alternative.
As the example stated above applying the average on those parameters without any filters in place do not provide any significant information. Therefore the averages are used together with additional characteristics gathered from the data set like the grouping of certain attributes. 

\section{Grouping Data}
One part of the statistical analysis is data grouping. One reason why data is grouped is to simplify complex data structures. Furthermore it enables the possibility to summarize certain characteristics present within the data set. Another benefit of grouping data is that it might reveal potential relationships. Additionally grouping can be used to improve predictive models as demonstrated in section\ref{sec:implementation}.
Analysing the given data set reveals the following attributes offer valuable insights when grouping is applied to them: 
\begin{itemize}
\item \verb|taskFrom_lat/lng| and \verb|taskTo_lat/lng|
\item \verb|createdAt|
\item \verb|taskFrom_time|
\end{itemize}
\subsection{Geographical Grouping}
As latitude and longitude are numerical described using numerical values their usage for geographical grouping is preferred over string values which are used for attributes like \verb|taskFrom_address| and \verb|taskTo_address|. Furthermore mathematical operations on coordinates like calculate the distance between two starting points allow us to modify how the data is grouped. Therefore the geographical data is utilized to group search requests by their departure and destination location. This provides the bus operator insights about popular routes. As bus operators determine maximal travel distances to departure places depending on their logistical base this data might reveal connections with high frequency. Depending on the additional distance the operator might need and it's current utilization an operator might decide to add an exception for this specific region to allocate additional bookings. Furthermore this information could be used to influence the pricing strategy for routes with high demand.
Geographical grouping in combination with the attribute \verb|amountSearchResults| can be applied to improve the offered service. Whenever \verb|amountSearchResults| is equal to zero no buses were offered for a certain request. As the grouping might reveal high demand for certain connections bus operators could consider to supply those connections as there are no competitors in this region. This results in a higher utilization of the operators bus fleet. 
\subsection{Grouping by Date}
Grouping search requests by date reveals valuable information that can be used for marketing purposes. Grouping request on an hourly basis reveals information about daytimes with high or low user frequency. This fact can be utilized to optimize potential ad campaigns. Grouping bookings by departure date reveals dates with high demand. Furthermore this metric is used to train the prediction models described in chapter \ref{chap:predict}. As this information indicates the availability of buses on the given day. Furthermore seasonal trends become visible.\newline

\section{Conversion Rates}




The implementation and visualisation of topics discussed in this section are carried out in chapter \ref{chap:analytical_dashboard}
\newline 
The carried out analysis does not claim for completeness. As there are several insights that can be gathered from the requests. The data set is analysed for metrics that provide insights that support operators as well as busfinder in their decision making and to further improve their service. 
%\include{kpi}
%----------------------------------------------------------------
%
%  File    :  prediction_model.tex
%
%  Authors : Thomas Lerchbaumer
% 
%  Created :  19 March 2022
% 
%  Changed :  19 March
% 
%----------------------------------------------------------------

\chapter{Predicting Future Bookings}
The knowledge of potential future bookings provide useful insights when it comes to yield management. Yield management in general describes controlling price and capacity control in a simultaneous ways \cite{yield_m}. Therefore those predictions can be used to support bus operators in their pricing strategy.  This chapter focuses on creating two prediction models utilizing different techniques based on the data that is available. 
Both models are implemented using python and the following libraries. 
\begin{itemize}
\item  \verb|matplotlib|\footnote{https://matplotlib.org/} - used for plotting
\item \verb|pandas|\footnote{https://pandas.pydata.org/} - used for data manipulation 
\item \verb|tesnorflow|\footnote{https://www.tensorflow.org/} - provides ML models
\item \verb|keras|\footnote{https://keras.io/} - Neural Network library
\end{itemize}

As there are various models available a literature review was conducted to figure out which models fit the purpose of time series forecasting. It turns out that the most promising NN that can be utilized for time series prediction are either Convolutional Neural Networks (CNN) or Recurrent Neural Networks (RNN) especially Long Short-Term Memory (LSTM)\cite{nn_1}\cite{nn_2}\cite{lstm_1}\cite{lstm_2}.

\section{The Models}
Both models CNN and RNN/LSTM can be used for time series forecasting. To create accurate prediction models a basic knowledge about models functionality is required. Therefore this section explains the components of each NN as well as the approaches those models follow. 

\subsection{LSTM}
LSTM is an RNN and was invented by \cite{lstm_inventor} in 1997. Until today this NN is widley used for time series forecasting and provides reliable results for short as well as long term predictions \cite{rnn_moharm}. LSTM have so called memory cells which are responsible to store the state of data. Whenever information arrives at a memory cell its outcome is defined by refreshing the cell state with the newly arrived information. LSTM utilizes gates to control a cells state by either including or excluding information \cite{lstm_stock}. Those gates are called: 
\begin{itemize}
\item input gate - data selection and storage for upcoming state
\item forget gate - data selection and storage which will not be used for the upcoming state
\item output gate - sets information within the state that is send to the output
\end{itemize}
Those gates are created by combining sigmoid functions. The results of this gates are values ranging from zero to one. A result of zero indicates the cell to not pass any infomration whereas values close to one indicates the cell to pass all information. 
The LSTM Module or Repeating module consits of four NN layers which interact together as shown in Figure \ref{fig:lstm_rep_model}.


\begin{figure}[H]
	\centering
		\includegraphics[width=14cm]{images/lstm_module}
	\caption{Repeating LSTM Module - [source:\cite{lstm_module}]}
	\label{fig:lstm_rep_model}
\end{figure}




\subsection{CNN}


\section{Implementation}


\section{Reliability Comparison - LSTM, CNN}

\section{Model accuracy}
Having a look at the model performance accuracy (comparing predictions of the model with already available data) , explain potential twerks that have been applied to the model itself to achieve a higher level of accuracy.


%----------------------------------------------------------------
%
%  File    :  analytical_dashboard.tex
%
%  Authors : Thomas Lerchbaumer
% 
%  Created :  19 March 2022
% 
%  Changed :  19 March
% 
%----------------------------------------------------------------

\chapter{Implementation of Averages and Grouping}
\label{chap:analytical_dashboard}
Since this paper aims to extract metrics that can contribute to an analytical dashboard this section focuses on the implementation of various groupings as well as their visualisation. As the analytical dashboard is web based a short overview of its technical setup is given in section \ref{sec:tech_set}.

\section{Technical Setup}
\label{sec:tech_set}
As those implementations will contribute to an analytical dashboard the visualisation of the prepared data is implemented using React. For each grouping a separate React component is created. Communication between the front- and backend is accomplished using a Rest interface. 

\begin{itemize}
\item  \verb|React|\footnote{https://react.dev/} - used for visualization
\item \verb|Python|\footnote{https://www.python.org/} - used to implement the grouping logic and interaction with the database
\item \verb|fastapi|\footnote{https://fastapi.tiangolo.com/} - Rest interface (Connection between backend and frontend) 
\item \verb|geopy|\footnote{https://geopy.readthedocs.io/en/stable/} - used for calculations on latitude and longitude 
\end{itemize}
\section{Grouping Implementation}

\subsection{Grouping by Geographical Attributes}
The dataset for bus search requests (after cleaning) currently has around 230.000 entries. As the geographical calculations as well as the grouping logic are computational expensive tasks it is not feasible to redo this logic on every request. To enable real time requests with various filters onto the grouped data an additional database table is introduced. The table consists of two attributes: 
\begin{itemize}
\item \verb|parent_id (FK - search_requests(task_id)| - parent that defines a region 
\item \verb|child_id (FK - search_requests(task_id)| - entries that belong to a certain parent region  
\end{itemize}
This design comes along with several advantages. The logic for analytical requests and the grouping logic can be separated. Furthermore new entries for groupings can be added to a potential parent in a more efficient way as those entries only need to be compared to the attributes of \verb|parent_id|.

For geographical grouping the following logic is applied:
\begin{lstlisting}
radius = 20
    res = collections.defaultdict(list)
    tmp_list = collections.defaultdict(list)
    for point_a in data:
        tmp_point_a_dep = tuple((point_a['taskFrom_lng'], point_a['taskFrom_lat']))
        tmp_point_a_dest = tuple((point_a['taskTo_lng'], point_a['taskTo_lat']))
        key = point_a['id']
        found = False
        for key_b in res:
            point_b = res[key_b][0]

            tmp_point_b_dep = tuple((point_b['taskFrom_lng'], point_b['taskFrom_lat']))
            tmp_point_b_dest = tuple((point_b['taskTo_lng'], point_b['taskTo_lat']))
            actual_distance_dep = distance.great_circle(tmp_point_a_dep, tmp_point_b_dep).km
            actual_distance_dest = distance.great_circle(tmp_point_a_dest, tmp_point_b_dest).km
            if actual_distance_dep < radius and actual_distance_dest < radius:
                res[key_b].append(point_a)
                found = True
                break
        if not found:
            res[key].append(point_a)
\end{lstlisting}
This logic performs the comparison between two geographical entries. Whenever \verb|point\_b| is within a 20km radius of the departure and destination area from \verb|point\_a| \verb|point\_b| is added to the group from \verb|point\_a|. Utilizing the demonstrated database design makes it possible to still include filters like date ranges to filter groupings by date. Furthermore the averages for attributes explained in section \ref{sec:averages} can be utilized and compared to evaluate certain trends. By adjusting the logic above and removing the constraint that both destination and departure has to be within a certain range this data can further be grouped into popular destination places as well as popular departure places. Depending on the grouping technique various insights can be gathered. By analysing grouped data that mark routes (same departure/destination radius) the pricing strategy for routes with high demand can be adapted. Whereas analysing the grouping on departure places with high traffic bus operators might consider to expand their maximum approaching distance. Furthermore by changing the way how the data is fetched from the database by querying results where no bus was offered \verb|amountSearchResults == 0| a different grouping becomes available for analysing. 
\subsubsection{Visualisation}
Depending on how the entries are grouped different visualization techniques are applied. To detect departure areas with high demand a heatmap is utilised. Therefore an interactive map is utilized. Onto this map areas with high demands are plotted following a color scheme. Red circles indicate high demand whereas blue areas indicate areas with less demand as shown in figure \ref{fig:heatmap_dep}. Furthermore this visualization can be filtered by applying date ranges. Additionally areas with high demand are displayed in a table including total count of requests for a certain area, average pax as well as the average travel distance.
\newline
As a heatmap is not suitable to visualise the grouped routes those groupings are displayed using a table. Additionally the table is filterable by date in order to be able to analyse the data split up into certain time periods. Furthermore the table hosts information about average pax and average travel distance. 
\begin{figure}[H]
	\centering
		\includegraphics[width=15cm]{images/heatmap_dep}
	\caption{Interactive heatmap highlighting departure places with high demand - [source:[author]]}
	\label{fig:heatmap_dep}
\end{figure}


\subsection{Grouping by Time}
In contrast to geographical grouping, groupings by time do not require a table within the database in order to process requests in real time. To group requests by time the following logic is applied to the fetched data: 
\begin{lstlisting}
def group_requests_by_time(data):
    df = pd.DataFrame(data)
    df = pd.to_datetime(df[0])
    grouped = df.groupby(df.dt.hour).count()
    for name in grouped.index:
        tmp = {"x": str(name) + ":00", "y": int(grouped.loc[name])}
        res[0]['data'].append(tmp)
    return res
\end{lstlisting}
To group search requests by days the following logic is applied: 
\begin{lstlisting}
def group_requests_by_day(data):
    data_format = pd.DataFrame(data)
    data_format = pd.to_datetime(data_format[0], format='%Y-%m-%d %H:%M:%S')
    grouped = data_format.groupby([data_format.dt.year, data_format.dt.month, data_format.dt.day]).count()
    res = []
    for name in grouped.index:
        date = str(name[0]) + "-" + str(name[1]) + "-" + str(name[2])
        tmp = {"value": int(grouped.loc[name]), "day": date}
        res.append(tmp)
    res = sorted(res, key=lambda x: x['day'])
    return res
\end{lstlisting}
Depending on which attribute e.g. \verb|createdAt| or \verb|taskFrom_time| is fetched from the database different groupings become available. Both attributes provide valuable information. When the data is grouped by utilising the attribute \verb|createdAt| the success rate of previous marketing campaigns can be evaluated. Furthermore this information can be consulted to plan future marketing campaigns. By grouping the data using the attribute \verb|taskFrom_time| seasonal patterns become visible as shown in \ref{fig:dep_grouping}
\subsubsection{Visualisation}
The hourly groupings are visualised using a line chart as demonstrated in figure \ref{fig:hourly_grouping}.
\begin{figure}[H]
	\centering
		\includegraphics[width=15cm]{images/hourly_grouping}
	\caption{Requests grouped on hourly basis- [source:[author]]}
	\label{fig:hourly_grouping}
\end{figure}
By analysing figure \ref{fig:hourly_grouping} the highest number of the most search requests are made at around 9 am. By hovering over each data point the amount of search requests become visible. 
\newline
 
\begin{figure}[H]
	\centering
		\includegraphics[width=15cm]{images/grouping_by_taskFrom}
	\caption{Daily grouping of departure date s- [source:[author]]}
	\label{fig:grouping_dep_daily}
\end{figure}
The grouping by day is visualized by a heatmap mapped to an calendar. By investigating the results for different years seasonal patterns become visible.
\begin{figure}[H]
	\centering
		\includegraphics[width=15cm]{images/daily_grouping_search_requets}
	\caption{Daily grouping of search requests - [source:[author]]}
	\label{fig:grouping_search_daily}
\end{figure}
All of the extracted groupings are filterable by date ranges in real time. Furthermore whenever a filter is applied the average pax as well as the average distance is provided to the user.
%----------------------------------------------------------------
%
%  File    :  available_dataset.tex
%
%  Authors : Thomas Lerchbaumer
% 
%  Created :  19 March 2022
% 
%  Changed :  19 March 2022
% 
%----------------------------------------------------------------

\chapter{Results}
\label{chap:results}
%----------------------------------------------------------------
%
%  File    :  available_dataset.tex
%
%  Authors : Thomas Lerchbaumer
% 
%  Created :  19 March 2022
% 
%  Changed :  19 March 2022
% 
%----------------------------------------------------------------

\chapter{Future work}
\label{chap:future_work}

% --- Bibliography ------------------------------------------------------

%IEEE Citation [1]
\bibliographystyle{IEEEtran}
%for alphanumeric citation eg.: [ABC19]
%\bibliographystyle{alpha}

% List references I definitely want in the bibliography,
% regardless of whether or not I cite them in the thesis.

\newpage
\addcontentsline{toc}{chapter}{Bibliography}
\bibliography{testBib}

\newpage

% --- List of Figures ----------------------------------------------------

\addcontentsline{toc}{chapter}{List of Figures}
\listoffigures


% --- List of Tables -----------------------------------------------------

\newpage
\addcontentsline{toc}{chapter}{List of Tables}
\listoftables

% --- Appendix A -----------------------------------------------------

\backmatter
\appendix
\begin{appendices}
%\chapter{Appendix}

%(Hier können Schaltpläne, Programme usw. eingefügt werden.)

\clearpage
\end{appendices}

\end{document}
