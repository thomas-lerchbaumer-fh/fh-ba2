%----------------------------------------------------------------
%
%  File    :  available_dataset.tex
%
%  Authors : Thomas Lerchbaumer
% 
%  Created :  19 March 2022
% 
%  Changed :  19 March 2022
% 
%----------------------------------------------------------------

\chapter{Results}
\label{chap:results}
The conducted analysis on search requests revealed several valuable insights ranging from averages used in a timely context to seasonal bookings patterns. Whereas averages can be conducted by bus operators for their future planing and decision making. By looking at averages and their development over time bus operators can use attributes like pax and travelled distance in meters for their future fleet planning and bus requirements. Furthermore the hourly grouping of search requests revealed peak hours. This information further can be used to optimize the scheduling of advertising campaigns. Bookings grouped by their date highlighted seasonal trends. As this seasonal trend is repeating throughout the years the knowledge can be used to optimize the pricing strategy. Moreover groupings of departure and destination regions revealed certain areas with high utilization. This fact can also be used to influence an operators pricing strategy as well as an operators decision to increase his initial range to start a journey. Also departure areas with low coverages in terms of bus offers are revealed by changing the initial request to the databases. All the information above can either be used to maximize an operators profit and also to support during data driven decision making. Furthermore the visualization of the metrics mentioned above support operators as well as the team from busfinder to derive certain trends and subsequently use the gathered insights during their decision making process. 
\newline
\newline