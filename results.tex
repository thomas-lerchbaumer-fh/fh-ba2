%----------------------------------------------------------------
%
%  File    :  available_dataset.tex
%
%  Authors : Thomas Lerchbaumer
% 
%  Created :  19 March 2022
% 
%  Changed :  19 March 2022
% 
%----------------------------------------------------------------

\chapter{Results}
\label{chap:results}
The conducted analysis on search requests revealed several valuable insights ranging from averages used in a timely context to seasonal bookings patterns. Whereas averages can be conducted by bus operators for their future planning and decision making. By looking at averages and their development over time bus operators can use attributes like pax and travelled distance in meters for their future fleet planning and bus requirements. Furthermore the hourly grouping of search requests revealed peak hours. This information further can be used to optimise the scheduling of advertising campaigns. Bookings grouped by their date highlighted seasonal trends. As this seasonal trend is repeating throughout the years the knowledge can be used to optimise the pricing strategy. Moreover groupings of departure and destination regions revealed certain areas with high utilisation. This fact can also be used to influence an operator's pricing strategy as well as an operator's decision to increase his initial range to start a journey. Also departure areas with low coverages in terms of bus offers are revealed by changing the initial request to the databases. All the information above can either be used to maximize an operator's profit and also to support during data driven decision making. Furthermore the visualisation of the metrics mentioned above support operators as well as the team from busfinder to derive certain trends and subsequently use the gathered insights during their decision making process. 
\newline
\newline
To avoid the occurrence of overfitting the lack of data during the pandemic was compensated utilising data augmentation. Furthermore the analysis on the available data highlighted seasonal trends which were further utilised during the process of feature engineering. 
The experiment carried out during chapter \ref{chap:predict} demonstrated that the available data suffices to create a prediction model capable of predicting the future capacity required of buses. The most accurate results were achieved by utilising a LSTM model. Whereas the predictions are not 100 \% accurate the model is capable of predicting peaks that occurred within the test set. Using the predicted values improves the current YM in place because future bookings can be assessed and therefore the pricing for high utilisation days can be adapted. As faulty predictions could lead to negative effects within a production environment security mechanisms need to be integrated. Furthermore continues training and model enhancements are required to ensure the models performance.