%----------------------------------------------------------------
%
%  File    :  introduction.tex
%
%  Authors : Thomas Lerchbaumer
% 
%  Created :  19 March 2022
% 
%  Changed :  19 March
% 
%----------------------------------------------------------------

\chapter{Introduction}
\label{chap:introduction}
The rise of Machine Learning (ML), and especially Neural Networks (NN), has created so far unheard-of capabilities for predictive analysis when looking at individual traffic in the modern transportation landscape. As digitalization emerged rapidly over the past years the possibilities for data-driven decision making further evolved as well. In general data-driven decision making is a process which involves collecting data to measure goals or Key Performance Indicators (KPIs). Furthermore data gathered through business processes can also be consulted in order to support business owners during their decision making process.
\newline
\newline
Predictive analysis involves statistical techniques and ML algorithms to analyse current as well as historical information in order to predict future events. The applications of predictive analysis span various sectors and industries. In finance it can be utilised to predict stock market trends \cite{stock_market}. When looking at healthcare it can assist in predicting diseases and to support doctors in their diagnosis \cite{health_care}. In retail it can help to understand customer behaviours \cite{retail}. How well certain algorithms can perform is influenced by the quality of available data \cite{data_qual}. In recent years ML techniques, especially NN, demonstrated promising results when it comes to time series forecasting. Certain NN models have the ability to detect non-linear relationship within a given dataset. Recurrent Neural Networks (RNN)including Long Short-Term Memory (LSTM) are commonly used approaches in this area as they have the ability to capture long term dependencies \cite{intro_ml_1}.
\newline
\newline
Furthermore each company has to measure their Key Performance Indicators (KPIs). Those KPIs reflect a company's short and long-term goals. Keeping track of them help companies to see how they perform against their goals and to carry out adjustments whenever needed.\cite{kpi_imrpove_businiess} To stay competitive, one crucial aspect is how this gathered data can be accessed and analysed. One very simple approach is to manually calculate defined KPIs when needed. As this process is rather time consuming a different approach is to provide Business Indicators Management (BIM) \cite{kpi_imrpove_decision_making}. Such systems are usually software based and lay the fundamentals for data driven decision making, as they come along with several advantages. They reduce the time required to analyse indicators which are necessary for decisions. Additionally various filters can be applied to data in real time which improves the agility when analysing data. As those analytical dashboards often use various visualisation methods ranging from simple plots to interactive maps unseen trends and insights become visible.\cite{kpi_imrpove_decision_making}
\newline
\newline 
One crucial element for every business is to stay ahead of their competitors. Depending on the industry a company operates in, those strategies may vary but the utilisation of data is one key element to successfully put strategies into practice.
In context of individual transportation one key factor is to continuously improve the offered service.****CITEHERE**** Therefore data gathered from a digital booking platform like busfinder can reveal valuable information like customer preferences, preferred travel times, common departure and destination places as well as seasonal booking patterns.
\newline
\newline
As busfinder acts as interface between bus operators and their clients considerations about both parties have to be made whenever improvements are set in place. One improvement that favours both parties is Yield Management. For customers the utilisation of YM results in a higher price flexibility and transparency. Whereas operators can use it to maximise their revenue and to improve their efficiency when it comes to resource utilisation.\cite{yield_m} As the offered service from busfinder includes the complete route planning as well as the whole booking and payment process a vast amount of data is gathered in order to process customer requests. Furthermore plain search requests that do no result in a booking are also stored and usable for analytical purposes.
\newline
\newline
Busfinder strives to optimise its offered service by utilising the available data described during chapter \ref{chap:available_dataset}. The aim of this paper is to figure out what NNs are suitable to predict future bus bookings in order to improve the current YM and which attributes of bus search requests can be utilised to contribute to an analytical dashboard. To carry out which NNs and models are the most promising for time series forecasting a literature review is conducted. Furthermore the evaluated models are implemented to measure their performance. Both the implementation as well as the literature review are discussed during chapter \ref{chap:predict}. Additionally the search requests are analysed for metrics that support operators that follow a data driven decision approach. The theory on how to conduct such an analysis was covered in the author's previous thesis. The gathered insights and their implementation are explained during chapter \ref{chap:analytical_dashboard}.


